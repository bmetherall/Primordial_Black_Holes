\documentclass[10pt]{article}

\usepackage{amsfonts}
\usepackage{amsmath}
\usepackage{geometry}
\usepackage{amsthm}
\usepackage{mathrsfs}
%\usepackage{xcolor,graphicx}
%\usepackage{wrapfig}
%\usepackage{subcaption}
%\usepackage{hyperref}

\newgeometry{margin=1in}
\setlength\parindent{0pt}

\DeclareMathOperator{\sgn}{sgn}
\DeclareMathOperator{\Heavi}{H}

\DeclareMathOperator{\Hsign}{\mathscr{H}}
\newcommand\Hank[2][]{{\left( \Hsign_{#1} #2 \right) }}

\newtheorem{theorem}{Theorem}[section]
\newtheorem{definition}[theorem]{Definition}
\newtheorem{corollary}[theorem]{Corollary}

\begin{document}

\section{Analytic Solution}

\subsection{Eigenfunctions of the Laplacian}

\subsection{Solving the Velocity Potential}
We must write the gravitational potential as an infinite sum of Bessel functions to match the form of $\varphi$, to do so, we take the Hankel transform,
\begin{align*}
\frac{\partial \Phi}{\partial t}\bigg|_{z=0} &=  \int_0^\infty \Hank{\frac{\partial \Phi}{\partial t}\bigg|_{z=0}}(k) J_0(kr) k \, dk \\
&= G m v^2 t \int_0^\infty \Hank{\frac{1}{(r^2 + v^2 t^2)^{3/2}}}(k) J_0(kr) k \, dk \footnote{hey} \\
&= G m v^2 t \int_0^\infty \frac{1}{|vt|} e^{-k|vt|} J_0(kr) k \, dk \\
&= G m v \sgn(t) \int_0^\infty e^{-kv|t|} J_0(kr) k \, dk.
\end{align*}
\footnotetext{Note that $\int_0^\infty \sqrt{r}(r^2 + v^2 t^2)^{-3/2} dr = \Gamma^2(3/4) (\pi v^3 t^3)^{-1/2} < \infty$ for $t \neq 0$ which is not concerning since this is true almost everywhere, and so the condition in Theorem ***** is satisfied.}
We can now substitute this into the differential equation and find $\varphi$,
\begin{align*}
\int_0^\infty J_0(kr)  \ddot{T}(t) dk + g \int_0^\infty k J_0(kr) T(t) dk + Gmv \int_0^\infty \sgn(t) e^{-kv|t|} J_0(kr)k \, dk = 0 \\
\int_0^\infty \left[ \frac{\ddot{T}(t)}{k} + gT(t) + Gmv \sgn(t) e^{-kv|t|} \right] J_0(kr) k \, dk = 0,
\end{align*}
but, this is nothing more than the Hankel transform of the differential equation for $T$. By taking the Hankel transform of both sides we can remove the integral,
\begin{align*}
\frac{\ddot{T}(t)}{k} + gT(t) + Gmv \sgn(t) e^{-kv|t|} = 0.
\end{align*}
Clearly, the homogeneous solution is $T(t) = A \cos(\omega_k t) + B \sin(\omega_k t)$ with $\omega_k^2 = gk$. The form of the differential equations suggests the form $T(t) = C e^{-kv|t|}$ for the particular solution. Substituting this in yields
\begin{align*}
C \left(k^2 v^2 \sgn^2(t) e^{-kv |t|} + gk e^{-kv |t|} \right) &+ Gmvk \sgn(t) e^{-kv|t|} = 0,
\end{align*}
giving
\begin{align*}
C &= \frac{-Gmvk \sgn(t)}{k^2v^2 + gk} \\
&= \frac{Gmv}{g} \frac{-\sgn(t)}{1+kv^2/g}
\end{align*}
as the coefficient, and,
\begin{align*}
T(t) = \frac{Gmv}{g} \frac{1}{1+kv^2/g} \left( -\sgn(t) e^{-kv|t|} \right) + A \cos(\omega_k t) + B \sin(\omega_k t)
\end{align*}
as the full time component of the velocity potential. We can now apply the boundary conditions to find $A$, and $B$. Physically, we expect $T(t) \in C^1(-\infty,\infty)$, furthermore, we only expect the sinusoidal terms to contribute at times greater than zero, thus,
\begin{align*}
T(t) &= \frac{Gmv}{g}\frac{1}{1+kv^2/g} \left(-\sgn(t)e^{-kv|t|} + 2H(t)\cos(\omega_k t) \right)dk, \text{ and,} \\
\varphi &= \frac{Gmv}{g} \int_0^\infty \frac{J_0(kr)e^{kz}}{1+kv^2/g} \left(-\sgn(t)e^{-kv|t|} + 2H(t)\cos(\omega_k t) \right)dk.
\end{align*}

\begin{align*}
\varphi = \int_0^\infty J_0(kr) e^{kz} T(t) dk \\
\left( \frac{\partial \varphi}{\partial t} + (g \eta + \Phi) \right) \bigg|_{z=0} = 0 \\
\left( \frac{\partial^2 \varphi}{\partial t^2} + g \frac{\partial \varphi}{\partial z} + \frac{\partial \Phi}{\partial t} \right) \bigg|_{z=0} = 0\\
\text{where }\Phi = \frac{-Gm}{\sqrt{r^2+(z-vt)^2}} \\
\varphi = \frac{Gmv}{g} \int_0^\infty \frac{J_0(kr)e^{kz}}{1+kv^2/g} \left(-\sgn(t)e^{-kv|t|} + 2 \Heavi (t)\cos(\omega_k t) \right)dk \\
\text{with } \omega_k^2=gk
\end{align*}



\end{document}
