\documentclass[12pt]{article}

\usepackage{amsfonts}
\usepackage{amsmath}
\usepackage{geometry}
\usepackage{amsthm}
\usepackage{mathrsfs}
\usepackage{xcolor,graphicx}
\usepackage{subcaption}
\usepackage{setspace}

\newgeometry{margin=1in}
\setlength\parindent{0pt}

\DeclareMathOperator{\sgn}{sgn}
\DeclareMathOperator{\Heavi}{H}

\DeclareMathOperator{\Hsign}{\mathscr{H}}
\newcommand\Hank[2][]{{\left( \Hsign_{#1} #2 \right) }}

\newtheorem{theorem}{Theorem}[section]
\newtheorem{definition}[theorem]{Definition}
\newtheorem{corollary}[theorem]{Corollary}

\begin{document}

\doublespacing
\linespread{2}

\section{Analytic Solution}

Typically, in fluid dynamics, the scalar function known as the velocity potential denoted by $\varphi$, is what is desired. Once the velocity potential is known, the system is solved because  $\overset{\rightharpoonup}\nabla \varphi = \bf{u}$, and in the case of a free surface, the deformation of the surface can readily be found from the velocity potential as well. The velocity potential will be found in the coming sub-chapters, and then the energy deposited into the neutron star will be calculated.

\subsection{Eigenfunctions of the Laplacian}

The first step in analytically solving for the velocity potential will be to find the eigenfunctions of Laplace's equation, $\nabla^2 \varphi = 0$, since the velocity potential is conservative. Since this is a partial differential equation we will assume the solution is the product of univariate functions, $\varphi = f(r) g(\theta) h(z) T(t)$. Expanding the Laplacian for cylindrical systems gives 
\begin{align*}
\frac{1}{r}\frac{\partial}{\partial r} \left( r \frac{\partial \varphi}{\partial r} \right) + \frac{1}{r^2} \left( \frac{\partial^2 \varphi}{\partial \theta^2} \right) + \frac{\partial^2 \varphi}{\partial z^2} = 0.
\end{align*}

The temporal component can be divided out since the Laplacian does not involve time, and will be found later on. Substituting in our assumed form, and dividing by $\varphi$, yields
\begin{align}
\label{eq:laplaciansub}
\frac{1}{fr}\frac{\partial}{\partial r}(rf') + \frac{1}{r^2}\frac{g''}{g} + \frac{h''}{h} = 0.
\end{align}

Where primes denote the derivative with respect to the function's only variable. Notice that the function $h(z)$ has been separated from both $f(r)$, and $g(\theta)$, and since each function is univariate, it must be the case that the last term is constant,
\begin{align*}
\frac{h''}{h} = k^2.
\end{align*}

We cleverly force this constant to be positive to satisfy the boundary conditions, namely, that $h(-\infty) = 0$. The most general solution to this is of course exponentials,
\begin{align*}
h(z) = Ae^{kz} + Be^{-kz}.
\end{align*}

Since the velocity potential has to vanish at negative infinity, this is simplified to
\begin{align}
\label{eq:h}
h(z) = e^{kz},
\end{align}

without loss of generality we can assume the constant is one, since it can be absorbed into $T(t)$. By substituting in $k^2$ and multiplying through by $r^2$, \eqref{eq:laplaciansub} becomes
\begin{align}
\label{eq:laplacenoh}
\frac{r}{f}\frac{\partial}{\partial r} \left( rf' \right) + k^2r^2 + \frac{g''}{g} = 0,
\end{align}

and once again, $g(\theta)$ has been separated from $f(r)$, and so the last term must be constant --
\begin{align*}
\frac{g''}{g} = -\mu^2.
\end{align*}

The constant is forced to be negative since $g(\theta)$ is expected to be periodic, and not exponential. Of course, the general solution to this differential equation is 
\begin{align*}
g(\theta) = A \sin(\mu \theta) + B \cos(\mu \theta).
\end{align*}

However, our system is cylindrically symmetric; there is no way to differentiate one value of $\theta$ to another, thus, it is expected that $\varphi$ has no $\theta$ dependence, and consequentially, $g(\theta)$ must be constant, the only way this is satisfied is if $\mu = 0$, so that
\begin{align}
\label{eq:g}
g(\theta) = 1.
\end{align}

By substituting $\mu = 0$ into \eqref{eq:laplacenoh}, and multiplying through by $f$ we obtain
\begin{align*}
r \frac{\partial}{\partial r}(rf') + (k^2r^2 - 0^2)f = 0,
\end{align*}

which indeed is Bessel's differential equation of order $0$. Therefore,
\begin{align*}
f(r) = A J_0(kr) + B Y_0(kr),
\end{align*}

but, once again, $B$ must equal zero since $Y_0(0)$ is not finite which is unphysical. The radial part thus simplifies to
\begin{align}
\label{eq:f}
f(r) = J_0(kr).
\end{align}

By combining \eqref{eq:h}, \eqref{eq:g}, and \eqref{eq:f} we find that the velocity potential is 
\begin{align*}
\varphi \propto e^{kz}T(t)J_0(kr),
\end{align*}

where $k$ is the eigenvalues. Clearly though, the only restriction on $k$ is that it must be a non-negative real number, all of which are a valid solution to Laplace's equation. Therefore, the total solution must be the sum of all of these, or since $k$ is valid within an interval, we have the integral form 
\begin{align}
\label{eq:phieigen}
\varphi = \int_0^\infty e^{kz}T(t)J_0(kr)dk.
\end{align}

The temporal component can be solved for using equations of fluid dynamics.

\subsection{Temporal Component of the Velocity Potential}

***Add the part about the fluid dynamics equation*** \\

We must write the gravitational potential as an infinite sum of Bessel functions to match the form of $\varphi$, to do so, we take the Hankel transform,
\begin{align*}
\frac{\partial \Phi}{\partial t}\bigg|_{z=0} &=  \int_0^\infty \Hank{\frac{\partial \Phi}{\partial t}\bigg|_{z=0}}(k) J_0(kr) k \, dk \\
&= G m v^2 t \int_0^\infty \Hank{\frac{1}{(r^2 + v^2 t^2)^{3/2}}}(k) J_0(kr) k \, dk \footnote{hey} \\
&= G m v^2 t \int_0^\infty \frac{1}{|vt|} e^{-k|vt|} J_0(kr) k \, dk \\
&= G m v \sgn(t) \int_0^\infty e^{-kv|t|} J_0(kr) k \, dk.
\end{align*}
\footnotetext{Note that $\int_0^\infty \sqrt{r}(r^2 + v^2 t^2)^{-3/2} dr = \Gamma^2(3/4) (\pi v^3 t^3)^{-1/2} < \infty$ for $t \neq 0$ which is not concerning since this is true almost everywhere, and so the condition in Theorem *** is satisfied.}
We can now substitute this into *** to obtain,
\begin{align*}
\int_0^\infty J_0(kr)  \ddot{T}(t) dk + g \int_0^\infty k J_0(kr) T(t) dk + Gmv \int_0^\infty \sgn(t) e^{-kv|t|} J_0(kr)k \, dk = 0 \\
\text{or, } \int_0^\infty \left[ \frac{\ddot{T}(t)}{k} + gT(t) + Gmv \sgn(t) e^{-kv|t|} \right] J_0(kr) k \, dk = 0,
\end{align*}
but, this is nothing more than the Hankel transform of the differential equation for $T$. By taking the Hankel transform of both sides we can remove the integral,
\begin{align*}
\frac{\ddot{T}(t)}{k} + gT(t) + Gmv \sgn(t) e^{-kv|t|} = 0.
\end{align*}
Clearly, the homogeneous solution is $T(t) = A \cos(\omega_k t) + B \sin(\omega_k t)$ with $\omega_k^2 = gk$. The form of the differential equations suggests the form $T(t) = C e^{-kv|t|}$ for the particular solution. Substituting this in yields
\begin{align*}
C \left(k^2 v^2 \sgn^2(t) e^{-kv |t|} + gk e^{-kv |t|} \right) &+ Gmvk \sgn(t) e^{-kv|t|} = 0,
\end{align*}
giving
\begin{align*}
C &= \frac{-Gmvk \sgn(t)}{k^2v^2 + gk} \\
&= \frac{Gmv}{g} \frac{-\sgn(t)}{1+kv^2/g}
\end{align*}
as the coefficient, and,
\begin{align*}
T(t) = \frac{Gmv}{g} \frac{1}{1+kv^2/g} \left( -\sgn(t) e^{-kv|t|} \right) + A \cos(\omega_k t) + B \sin(\omega_k t)
\end{align*}
as the full time component of the velocity potential. We can now apply the boundary conditions to find $A$, and $B$. Physically, we expect $T(t) \in C^1(-\infty,\infty)$, furthermore, we only expect the sinusoidal terms to contribute at times greater than zero, thus,
\begin{align*}
T(t) &= \frac{Gmv}{g}\frac{1}{1+kv^2/g} \left(-\sgn(t)e^{-kv|t|} + 2H(t)\cos(\omega_k t) \right)dk, \text{ and,} \\
\varphi &= \frac{Gmv}{g} \int_0^\infty \frac{J_0(kr)e^{kz}}{1+kv^2/g} \left(-\sgn(t)e^{-kv|t|} + 2H(t)\cos(\omega_k t) \right)dk.
\end{align*}

\begin{align*}
\varphi = \int_0^\infty J_0(kr) e^{kz} T(t) dk \\
\left( \frac{\partial \varphi}{\partial t} + (g \eta + \Phi) \right) \bigg|_{z=0} = 0 \\
\left( \frac{\partial^2 \varphi}{\partial t^2} + g \frac{\partial \varphi}{\partial z} + \frac{\partial \Phi}{\partial t} \right) \bigg|_{z=0} = 0\\
\text{where }\Phi = \frac{-Gm}{\sqrt{r^2+(z-vt)^2}} \\
\varphi = \frac{Gmv}{g} \int_0^\infty \frac{J_0(kr)e^{kz}}{1+kv^2/g} \left(-\sgn(t)e^{-kv|t|} + 2 \Heavi (t)\cos(\omega_k t) \right)dk \\
\text{with } \omega_k^2=gk
\end{align*}

\newpage

\begin{figure}[p]
\begin{centering}
 \begin{subfigure}{\textwidth}
  \input{./Analytic000}
  \caption{$t=-1$ s.}
 \end{subfigure} \\
 \begin{subfigure}{\textwidth}
  \input{./Analytic001}
  \caption{$t=0$ s.}
 \end{subfigure} \\
 \begin{subfigure}{\textwidth}
  \input{./Analytic002}
  \caption{$t=1$ s.}
 \end{subfigure}
\end{centering}
\end{figure}

\begin{figure}[p] \ContinuedFloat
\begin{centering}
 \begin{subfigure}{\textwidth}
  \input{./Analytic003}
  \caption{$t=2$ s.}
 \end{subfigure} \\
 \begin{subfigure}{\textwidth}
  \input{./Analytic004}
  \caption{$t=3$ s.}
 \end{subfigure} \\
  \begin{subfigure}{\textwidth}
  \input{./Analytic005}
  \caption{$t=4$ s.}
 \end{subfigure}
 \end{centering}
\end{figure}




\end{document}
