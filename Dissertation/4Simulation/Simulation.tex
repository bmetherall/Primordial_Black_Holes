\documentclass[10pt]{article}

\usepackage{amsfonts}
\usepackage{amsmath}
\usepackage{geometry}
\usepackage{listings}
\usepackage[dvipsnames]{xcolor}
\usepackage{graphicx}
%\usepackage{wrapfig}
\usepackage{subcaption}
%\usepackage{hyperref}

\definecolor{codegreen}{rgb}{0,0.6,0}
\definecolor{codegray}{rgb}{0.5,0.5,0.5}
\definecolor{codepurple}{rgb}{0.58,0,0.82}
\definecolor{backcolour}{rgb}{0.95,0.95,0.92}
 
\lstdefinestyle{mystyle}{
    emph={self, Equation, loop, BlackHole2D,__init__, initialize, BlackHole, Application, Group},
    emphstyle=\color{PineGreen},
    commentstyle=\color{blue},
    keywordstyle={\color{BrickRed}\bfseries},
    numberstyle=\color{codegray},
    stringstyle=\color{codepurple},
    breakatwhitespace=false,         
    breaklines=true,                 
    captionpos=b,                    
    keepspaces=true,                 
    numbers=left,                    
    numbersep=5pt,                  
    showspaces=false,                
    showstringspaces=false,
    showtabs=false,                  
    tabsize=2
}

\newgeometry{margin=1in}
\setlength\parindent{0pt}

\begin{document}

\lstset{style=mystyle}

\section{Code}

\begin{figure}[htbp]
\lstinputlisting[language=Python, lastline=18]{../../BlackHoleEquation.py}
\caption{Class for adding the acceleration due to the primordial black hole.}
\label{fig:blackholeclass}
\end{figure}

\begin{figure}[htbp]
  \lstinputlisting[language=Python, firstnumber=19, firstline=19, lastline=21]{../../blackhole.py}
  \lstinputlisting[language=Python, firstnumber=40, firstline=40, lastline=59]{../../blackhole.py}
  \lstinputlisting[language=Python, firstnumber=82, firstline=82, lastline=82]{../../blackhole.py}
  \lstinputlisting[language=Python, firstnumber=171, firstline=171, lastline=173]{../../blackhole.py}
  \lstinputlisting[language=Python, firstnumber=194, firstline=194, lastline=195]{../../blackhole.py}
  \lstinputlisting[language=Python, firstnumber=212, firstline=212, lastline=217]{../../blackhole.py}
 \caption{Modifications of the hydrostatic tank.}
\end{figure}

\end{document}
