\chapter{Conclusions}
\label{chap:conc}

In this thesis we looked at the effect a primordial black hole has on the surface of a neutron star if the primordial black hole were to pass through the neutron star. To simplify the problem we first assume that a neutron star is an infinite plane, and that primordial black holes are point masses. One of the main results was analytically showing that the energy transfer in this model is
\begin{align*}
E &= 4 \pi \rho \frac{G^2 m^2}{g}.
\end{align*}
We also independently simulated the collision with smoothed particle hydrodynamics. The profile of the energy transfer matched similarly in both cases, however, in the simulation there was several disturbances in the energy skewing the shape. \\

If we were to further develop this work, there are a few obvious modifications, and extensions. Firstly, there are better methods of simulating the collision so that the fluid is started at equilibrium to stop the oscillations of the surface, and so the walls of the tank do not reflect the waves back into the centre. The next step to improve the result would be moving from an infinite plane to a spherical star, to better match reality. Defillon, Granet, Tinyakov, and Tytgat \cite{tidalcapture} also investigated the spherical case as well as the flat star, however, within the time constraints of this thesis, it was not looked at in great detail. \\

Finally, to obtain the most precise estimates of the energy transfer between a primordial black hole and a neutron star -- and what may be observed here on Earth -- much more sophisticated physics needs to be incorporated. For example, a group in Illinois uses full general relativistic magnetohydrodynamics (GRMHD) code to simulate events such as black hole mergers, or gravitational waves \cite{nuts}.