\documentclass[10pt]{article}

\usepackage{amsfonts}
\usepackage{amsmath}
\usepackage{geometry}
\usepackage{amsthm}
\usepackage{mathrsfs}
%\usepackage{xcolor,graphicx}
%\usepackage{wrapfig}
%\usepackage{subcaption}
%\usepackage{hyperref}

\newgeometry{margin=1in}
\setlength\parindent{0pt}

\DeclareMathOperator{\sgn}{sgn}
\DeclareMathOperator{\Heavi}{H}

\DeclareMathOperator{\Hsign}{\mathscr{H}}
\newcommand\Hank[2][0]{{(\Hsign_{#1} #2 ) }}

\newtheorem{theorem}{Theorem}[section]
\newtheorem{definition}[theorem]{Definition}
\newtheorem{corollary}[theorem]{Corollary}

\begin{document}

\section{Operators and Integral Transforms}
\begin{definition}[Operator]
\label{def:operator}
Let $A$, and $B$ be vector spaces with respective subspaces, $X$, and $Y$. An operator $\mathcal{T}$, maps any $x \in X$ to $Y$, and is denoted by $\mathcal{T}(x)$.
\end{definition}
Common examples of operators are the Sturm-Liouville operator, the Laplacian, or Hamiltonian. Our focus will be on the integral operator, or transform. Let the domain, and co-domain of the transform be $C[a,b]$ and $K: \mathbb{R}^2 \rightarrow \mathbb{R}$, then we can define our operator $\mathcal{T}:C[a,b] \rightarrow C[a,b]$ as
\begin{align*}
(\mathcal{T}f)(x) = \int_a^b f(y) K(x,y)dy,
\end{align*}
where $K$ is called the kernel function.

\begin{theorem}
\label{thm:continuity}
$\mathcal{T}f$ is continuous iff $\int_a^b |f(y)| dy < \infty$, and $K(x,y)$ is uniformly continuous on $[a,b]$.
\end{theorem}
\begin{proof}
For all $\epsilon > 0$, choose $\delta : |x - x_0| < \delta$, so that $|K(x,y) - K(x_0,y)| < \epsilon/M$, with $M = \int_a^b |f(y)| dy$. $\mathcal{T}f$ is continuous iff
\begin{align*}
|(\mathcal{T}f)(x) - (\mathcal{T}f)(x_0)| &= \left| \int_a^b K(x,y)f(y) dy - \int_a^b K(x_0,y)f(y) dy \right| \\
&\leq \int_a^b |K(x,y) - K(x_0,y)||f(y)| dy \\
&< \int_a^b \frac{\epsilon}{M} |f(y)| dy \\
&< \epsilon
\end{align*}
\end{proof}

\begin{corollary}
The conditions for Theorem \ref{thm:continuity} are satisfied if $a$ and $b$ are finite, as well as if $f$ and $K$ are bounded and continuous.
\end{corollary}
\begin{proof}
A bounded continuous function is integrable over a compact domain.
\end{proof}

\subsection{Hankel Transform}

\begin{definition}[Hankel Transform]
\label{def:hanktrans}
The Hankel transform of a function $f(s)$ is given by
\begin{align*}
\Hank[\nu]{f}(\sigma) = \int_0^\infty f(s) J_\nu(s \sigma) s \, ds,
\end{align*}
where $J_\nu$ is the Bessel function of the first kind, of order $\nu \geq -\frac{1}{2}$, and $\sigma$ is a non-negative real variable.
\end{definition}

\begin{corollary}[Inverse Hankel Transform]
\label{def:invhanktrans}
The Hankel transform is self-reciprocal, that is, the inverse Hankel transform is also given by Definition \ref{def:hanktrans}.
\end{corollary}
\begin{proof}
The Hankel transform is self-reciprocal
\begin{align*}
\iff f(s) &= \int_0^\infty \Hank[\nu]{f}(\sigma) J_\nu(s \sigma) \sigma \, d\sigma \\
\iff f(s) &= \int_0^\infty \int_0^\infty f(s') J_\nu(s \sigma) s' \, ds' J_\nu(s \sigma) \sigma \, d\sigma \\
&= \int_0^\infty f(s') s' \int_0^\infty J_\nu(s' \sigma) J_\nu(s \sigma) \sigma \, d\sigma \, ds' \\
&= f(s),
\end{align*}
by the orthogonality of the Bessel functions.
\end{proof}

\newpage

\begin{align*}
\varphi = \int_0^\infty J_0(kr) e^{kz} T(t) dk \\
\left( \frac{\partial \varphi}{\partial t} + (g \eta + \Phi) \right) \bigg|_{z=0} = 0 \\
\left( \frac{\partial^2 \varphi}{\partial t^2} + g \frac{\partial \varphi}{\partial z} + \frac{\partial \Phi}{\partial t} \right) \bigg|_{z=0} = 0\\
\text{where }\Phi = \frac{-Gm}{\sqrt{r^2+(z-vt)^2}} \\
\varphi = \frac{Gmv}{g} \int_0^\infty \frac{J_0(kr)e^{kz}}{1+kv^2/g} \left(-\sgn(t)e^{-kv|t|} + 2 \Heavi (t)\cos(\omega_k t) \right)dk \\
\text{with } \omega_k^2=gk
\end{align*}

We must write the gravitational potential as an infinite sum of Bessel functions,
\begin{align*}
\frac{\partial \Phi}{\partial t}\bigg|_{z=0} &= \int_0^\infty a(k; t)J_0(k r)k \, dk \\
a(k; t) &= \int_0^\infty \frac{\partial \Phi}{\partial t}\bigg|_{z=0} J_0(k r) r \, dr \text{ *HT*} \\
a(k;t) &= Gmv^2tk \int_0^\infty \frac{ J_0(k r) r}{(r^2+v^2t^2)^{3/2}} dr \\
&= Gmv^2t \frac{1}{|vt|} e^{-k|vt|} \text{ *HT*} \\
&= Gmv \sgn(t)e^{-kv|t|}
\end{align*}
\begin{gather*}
\int_0^\infty J_0(kr)  \ddot{T}(t) dk + g \int_0^\infty k J_0(kr) T(t) dk + Gmv \int_0^\infty \sgn(t) e^{-kv|t|} J_0(kr)k \, dk = 0 \\
\int_0^\infty \left[ \frac{\ddot{T}(t)}{k} + gT(t) + Gmv \sgn(t) e^{-kv|t|} \right] J_0(kr) k \, dk = 0 \\
\frac{\ddot{T}(t)}{k} + gT(t) + Gmv \sgn(t) e^{-kv|t|} = \int_0^\infty 0 \times J_0(kr) r \, dr = 0 \text{ *HT*} \\
T(t) = A \cos(\omega_k t) + B \sin(\omega_k t) \text{ homogenous solution } t \geq 0 \\
\text{Assume } T(t) = C e^{\xi |t|} \text{ for the particular solution} \\
\implies C \left(\xi^2 \sgn^2(t) e^{\xi |t|} + gk e^{\xi |t|} \right) + Gmvk \sgn(t) e^{-kv|t|} = 0 \\
\implies \xi = -kv, \quad C = \frac{-Gmvk \sgn(t)}{k^2v^2 + gk} = \frac{Gmv}{g} \frac{-\sgn(t)}{1+kv^2/g} \\
\implies T(t) = \frac{Gmv}{g} \frac{1}{1+kv^2/g} \left( -\sgn(t) e^{-kv|t|} + \Heavi(t) \left( \widetilde{A} \cos(\omega_k t) + \widetilde{B} \sin(\omega_k t \right) \right) \\
T(t) \text{ must be at least of class } C^1 \implies \widetilde{A} = 2, \quad \widetilde{B} = 0 \\
\implies \varphi = \frac{Gmv}{g} \int_0^\infty \frac{J_0(kr)e^{kz}}{1+kv^2/g} \left(-\sgn(t)e^{-kv|t|} + 2H(t)\cos(\omega_k t) \right)dk
\end{gather*}













\end{document}
