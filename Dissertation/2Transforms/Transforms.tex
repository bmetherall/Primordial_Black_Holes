\documentclass[10pt]{article}

\usepackage{amsfonts}
\usepackage{amsmath}
\usepackage{geometry}
\usepackage{amsthm}
\usepackage{mathrsfs}
%\usepackage{xcolor,graphicx}
%\usepackage{wrapfig}
%\usepackage{subcaption}
%\usepackage{hyperref}

\newgeometry{margin=1in}
\setlength\parindent{0pt}

\DeclareMathOperator{\sgn}{sgn}
\DeclareMathOperator{\Heavi}{H}

\DeclareMathOperator{\Hsign}{\mathscr{H}}
\newcommand\Hank[2][]{{\left( \Hsign_{#1} #2 \right) }}

\newtheorem{theorem}{Theorem}[section]
\newtheorem{definition}[theorem]{Definition}
\newtheorem{corollary}[theorem]{Corollary}

\begin{document}

\section{Operators and Integral Transforms}
\begin{definition}[Operator]
\label{def:operator}
Let $A$, and $B$ be vector spaces with respective subspaces, $X$, and $Y$. An operator $\mathcal{T}$, maps any $x \in X$ to $Y$, and is denoted by $\mathcal{T}(x)$.
\end{definition}
Common examples of operators are the Sturm-Liouville operator, the Laplacian, or Hamiltonian. Our focus will be on the integral operator, or transform. Let the domain, and co-domain of the transform be $C[a,b]$ and $K: \mathbb{R}^2 \rightarrow \mathbb{R}$, then we can define our operator $\mathcal{T}:C[a,b] \rightarrow C[a,b]$ as
\begin{align*}
(\mathcal{T}f)(x) = \int_a^b f(y) K(x,y)dy,
\end{align*}
where $K$ is called the kernel function.

\begin{theorem}
\label{thm:continuity}
$\mathcal{T}f$ is continuous if $\int_a^b |f(y)| dy < \infty$, and $K(x,y)$ is uniformly continuous on $[a,b]$.
\end{theorem}
\begin{proof}
For all $\varepsilon > 0$, choose $\delta : |x - x_0| < \delta$, so that $|K(x,y) - K(x_0,y)| < \varepsilon/M$, with $M = \int_a^b |f(y)| dy$. Then,
\begin{align*}
|(\mathcal{T}f)(x) - (\mathcal{T}f)(x_0)| &= \left| \int_a^b K(x,y)f(y) dy - \int_a^b K(x_0,y)f(y) dy \right| \\
&\leq \int_a^b |K(x,y) - K(x_0,y)||f(y)| dy \\
&< \int_a^b \frac{\varepsilon}{M} |f(y)| dy \\
&< \varepsilon,
\end{align*}
and $\mathcal{T}f$ is continuous.
\end{proof}

The conditions for Theorem \ref{thm:continuity} are automatically satisfied if $a$ and $b$ are finite, in addition to $K$ being bounded and continuous. A bounded continuous function over a compact domain is uniformly continuous and integrable.

\subsection{Hankel Transform}

\begin{definition}[Hankel Transform]
\label{def:hanktrans}
The Hankel transform of a function $f(s)$ is given by
\begin{align*}
\Hank[\nu]{f}(\sigma) = \int_0^\infty f(s) J_\nu(s \sigma) s \, ds,
\end{align*}
where $J_\nu$ is the Bessel function of the first kind, of order $\nu \geq -\frac{1}{2}$, and $\sigma$ is a non-negative real variable.
\end{definition}
Notice though, that the kernel function of the Hankel transform is not $J_\nu (s \sigma) s$, but in fact $J_\nu (s \sigma) \sqrt{s}$, the other $\sqrt{s}$ factor gets absorbed into $f$. This is to ensure uniform continuity of the kernel, and the continuity of the transform. As a consequence, we have the modified condition that $\int_0^\infty \sqrt{s}|f(s)| ds < \infty$. **Include in appendix?

\begin{corollary}[Inverse Hankel Transform]
\label{def:invhanktrans}
The Hankel transform is self-reciprocal, that is, the inverse Hankel transform is also given by Definition \ref{def:hanktrans}.
\end{corollary}
\begin{proof}
The Hankel transform is self-reciprocal
\begin{align*}
\iff f(s) &= \int_0^\infty \Hank[\nu]{f}(\sigma) J_\nu(s \sigma) \sigma \, d\sigma \\
\iff &= \int_0^\infty \int_0^\infty f(s') J_\nu(s \sigma) s' \, ds' J_\nu(s \sigma) \sigma \, d\sigma \\
&= \int_0^\infty f(s') s' \int_0^\infty J_\nu(s' \sigma) J_\nu(s \sigma) \sigma \, d\sigma \, ds' \\
&= f(s),
\end{align*}
by the orthogonality of the Bessel functions.
\end{proof}


\end{document}
