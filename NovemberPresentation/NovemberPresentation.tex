\documentclass[mathserif]{beamer}
\usepackage{amsmath}
\usepackage{color}
\usepackage{amsfonts}
\usepackage{xcolor,graphicx}
\usepackage{geometry}
\usepackage{hyperref}
\usepackage{multimedia}

\title{Collisions of Primordial Black Holes and Neutron Stars}
\author[Brady Metherall]{Brady Metherall \and 100516905}
\date{Monday November 28, 2016}

\usetheme[hideothersubsections]{Berkeley}
\usecolortheme{whale}

\DeclareMathOperator{\sgn}{sgn}

\begin{document}
\frame{\titlepage}
\setlength\parindent{0pt}

\section{Introduction}

\frame{
\frametitle{Introduction}
A primordial black hole is a black hole that was hypothetically created in the early universe.
}

\section{Flat Star Model}

\frame{
\frametitle{Flat Star Model}
\begin{itemize}
\item Neutron stars are flat and infinite
\item Primordial black holes are point masses
\item Neutron stars are incompressible fluids
\item Gravitational interactions are Newtonian
\item Constant velocity
\end{itemize}
}

\subsection{Solving for $\varphi$}

\frame{
\frametitle{Eigenfunctions of Laplacian}
Assume a product solution for the velocity potential and solve the Laplacian.
\begin{align*}
\varphi &= R(r) Z(z) \Theta(\theta) T(t) \\
\nabla^2 \varphi &= 0 \\
\implies \varphi &\propto \left\{ \begin{matrix}J_\mu(kr) \\ Y_\mu(kr) \end{matrix} \right\}  \left\{ \begin{matrix}e^{-kz} \\ e^{kz} \end{matrix} \right\} \left\{ \begin{matrix}\sin(\mu \theta) \\ \cos(\mu \theta) \end{matrix} \right\} T(t) \\
\implies \varphi &\propto J_0(kr) e^{kz} T(t)
\end{align*}
$T(t)$ comes from boundary conditions.
}
\frame{
\frametitle{Solving for $\varphi$}
\begin{gather*}
\Phi = \frac{-Gm}{\sqrt{r^2+(z+vt)^2}} \\
\left( \frac{\partial^2 \varphi}{\partial t^2} + g \frac{\partial \varphi}{\partial z} \right) \bigg|_{z=0} = - \frac{\partial \Phi}{\partial t} \bigg|_{z=0} \\
\varphi = \frac{Gmv}{g} \int_0^\infty \frac{J_0(kr)e^{kz}}{1+kv^2/g} \left(-\sgn(t)e^{-kv|t|} + 2H(t)\cos(\omega_k t) \right)dk \\
\text{with } \omega_k^2=gk
\end{gather*}
}

\subsection{Solving for $\eta$}

\frame{
\frametitle{Deformation of Surface}
\begin{gather*}
\frac{\partial \varphi}{\partial z} \bigg|_{z=0} = \frac{\partial \eta}{\partial t} \\
\eta = \frac{Gm}{g} \int_0^\infty \frac{J_0(kr)}{1+kv^2/g} \left(e^{-kv|t|} + 2H(t)v \sqrt{\frac{k}{g}} \sin(\omega_k t) \right)dk
\end{gather*}
}

\subsection{Plotting $\eta$}

\frame{
\frametitle{Plotting $\eta$}
\begin{figure}
\begin{centering}
  \input{../Wave/Images/plot000}
  \caption{$t=-1$ s.}
\end{centering}
\end{figure}
}
\frame{
\frametitle{Plotting $\eta$}
\begin{figure}
\begin{centering}
  \input{../Wave/Images/plot001}
  \caption{$t=0$ s.}
\end{centering}
\end{figure}
}
\frame{
\frametitle{Plotting $\eta$}
\begin{figure}
\begin{centering}
  \input{../Wave/Images/plot002}
  \caption{$t=1$ s.}
\end{centering}
\end{figure}
}
\frame{
\frametitle{Plotting $\eta$}
\begin{figure}
\begin{centering}
  \input{../Wave/Images/plot003}
  \caption{$t=2$ s.}
\end{centering}
\end{figure}
}
\frame{
\frametitle{Plotting $\eta$}
\begin{figure}
\begin{centering}
  \input{../Wave/Images/plot004}
  \caption{$t=3$ s.}
\end{centering}
\end{figure}
}
\frame{
\frametitle{Plotting $\eta$}
\begin{figure}
\begin{centering}
  \input{../Wave/Images/plot005}
  \caption{$t=4$ s.}
\end{centering}
\end{figure}
}

\subsection{Solving for $E$}

\frame{
\frametitle{Energy Transferred}
\begin{align*}
E &= \lim_{t \rightarrow \infty} \frac{1}{2}\rho \int_{-\infty}^0 \int_0^\infty |\nabla \varphi |^2 \, r dr dz \int_0^{2\pi} d\theta \\
& \qquad + \rho g \int_0^\infty \int_0^\eta z dz \, rdr \int_0^{2\pi} d\theta \\
&= 4 \pi \rho \frac{G^2 m^2}{g}
\end{align*}
}

\section{Numerical Approach}

\subsection{Smooth Particle Hydrodynamics}

\frame{}

\end{document}